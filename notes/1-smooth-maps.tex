\begin{notes}
      Definitions
\end{notes}
\hrule

\begin{enumerate}
      \item A function $f:X\to\rom$ $(X\subset\ron)$ is \underline{smooth} if around each $x\in X$ there is an open set
            $U\subset\ron$ and a smooth map $F:U\to\rom$ such that $F$ equals $f$ on $U\cap X$.
      \item A smooth map $f:X\to Y$ of subsets of Euclidean spaces is called \underline{diffeomorphism} if it is bijective
            and if the inverse map $f^{-1}:Y\to X$ is smooth.
      \item A set $X\subset\ron$ is a $k$-dimensional manifold if every $x\in X$ posseses a neighborhood $V$ which is
            diffeomorphic to an open set $U\subset\rl^k$. The diffeomorphism $\varphi:U\to V$ is called a
            \underline{parametrization} of the neighborhood $V$ and the inverse diffeomorphism $\varphi^{-1}:V\to U$ is
            called a \underline{coordinate system} on $V$.
      \item (Problem 3)
            \begin{enumerate}
                  \item[i)] For every $x\in X$, there exists an open set $U\subset\ron$ and a smooth map $F:U\to Y$ such that
                        $F|_{U\cap X}=f$.
                  \item[ii)] For every $f(x)\in Y$, there exists an open set $V\subset\rom$ and a smooth map $G:V\to Z$ such that
                        $G|_{V\cap Y}=g$.
            \end{enumerate}
\end{enumerate}